\documentclass{article}

\usepackage{amsmath, amssymb} %standard packages for math writing
%\usepackage{mathpazo} % a better font than the default one
%\usepackage{amsthm}
\usepackage{bm} %bold math
\usepackage{upgreek} %add special greek symbols
\usepackage{bbold}
\usepackage{color}
\usepackage{geometry}
\usepackage{graphicx}%include figure files
\usepackage{float}
\usepackage{ntheorem}
\usepackage[most]{tcolorbox}
\usepackage[colorlinks, linkcolor=blue, anchorcolor=blue, citecolor=blue, urlcolor=blue]{hyperref} % add hypertext
\usepackage[perpage,symbol*]{footmisc}

\geometry{a4paper, centering, scale=0.8}

\def\RR{\mathbb{R}}
\def\dd{\mathrm{d}}
\def\1{\mathbb{1}}

\newcommand{\ww}[1]{\textcolor{red}{\bf~[Notice:~#1]~}}	% My comments
\newcommand{\referee}[1]{\textcolor{blue}{\textit{#1~\\}}}	% referee comments
\newcommand{\vv}{\boldsymbol}

\newenvironment{proof}{\emph{Proof.}}{\hfill $\square$\par}

\theoremindent1.4em
\newtheorem{definition}[subsection]{Definition}

\definecolor{browna}{rgb}{0.76,0.72,0.65}
\definecolor{brownb}{rgb}{0.71,0.69,0.65}

\newtcbtheorem[no counter]{theorem}{Theorem}{
  enhanced,
  sharp corners,
  attach boxed title to top left={
    yshifttext=-1mm
  },
  colback=white,
  colframe=browna,
  fonttitle=\bfseries,
  boxed title style={
    sharp corners,
    size=small,
    colback=browna,
    colframe=browna,
  } 
}{thm}

\newtcbtheorem[number within=section]{example}{Example}{
  enhanced,
  sharp corners,
  attach boxed title to top left={
    yshifttext=-1mm
  },
  colback=white,
  colframe=brownb,
  fonttitle=\bfseries,
  boxed title style={
    sharp corners,
    size=small,
    colback=brownb,
    colframe=brownb,
  } 
}{exm}

\definecolor{backblue}{rgb}{0.84,0.91,0.94}
\definecolor{leftblue}{rgb}{0,0.40,0.62}

\newtcbtheorem{prop}{Proposition}{
  coltitle=black,
  colback=backblue,
  colframe=leftblue,
  fonttitle=\bfseries,
  detach title,
  boxrule=-0.1pt,
  leftrule=2pt,
  attach title to upper,
  sharp corners,
  left=3mm,
}{claim}

%
%  Created by WW on 05/25/21
%  Copyright © WW. All rights reserved.
%
\begin{document}
\section{Templates for theorems}
\hangafter 1
\hangindent 1.7em
\noindent 1. The usual theorem environment is like:
\begin{definition}
    $\1$ is unit matrix.
\end{definition}

\hangafter 1
\hangindent 1.7em
\indent we can contruct a better one
\begin{theorem}[label=th:2]{Newton's second law}{this is some notes about the theorem which won't show up}
    Equation of motion:
    $$\vv F=m\vv a,$$
    where $m$ is the mass, $\vv F$ is the force, and $\vv a$ is the acceleration.
\end{theorem}
or
\begin{theorem}{1.1}{}
    \textbf{Newton's second law.} Equation of motion:
    $$\vv F=m\vv a,$$
    where $m$ is the mass, $\vv F$ is the force, and $\vv a$ is the acceleration.
\end{theorem}

\section{Templates for examples}
\hangafter 1
\hangindent 1.7em
\indent 1. Set \textbf{[number within=section]} in the preamble, and use package \textbf{hyperref}.
\begin{example}[label=ex:1]{This is tilte}{}
    We can see that this example was numbered automatically. 
\end{example}
We can label and cite Example \ref{ex:1} in this way. 
\begin{example}{}{}
    The title can also be left empty. Note that the `:' vanished magically.
\end{example}
\begin{example*}{}{}
    This theorem environment is not numbered by adding a *.
\end{example*}
we can also adjust the indentaion:
\begin{example*}[before={\vspace{0.5em}\par\parindent=1.2em}, width=\linewidth-1.2em]{}{}
    [before=\{$\backslash$vspace\{0.5em\}$\backslash$par$\backslash$parindent=1.2em\}, width=$\backslash$linewidth-1.2em]
\end{example*}

\section{Templates for claims}
\hangafter 1
\hangindent 1.7em
\indent 1. Another beautiful theorem environment\footnote{This is for foot notes.}:
\begin{prop*}[before={\par\parindent=1.2em}, width=\linewidth-1.2em]{1}{}
    \begin{equation*}
        I\equiv\int_0^\infty\dd x~e^{ikx}=\frac{i}{k+i\epsilon}=\mathcal{P}\frac{i}{k}+\pi\delta(k),
    \end{equation*}
    where $\epsilon>0$.
\end{prop*}
\begin{proof}
    To perform the integral, add in a small convergence factor, $e^{-\epsilon x}$ with $\epsilon>0$,
    \begin{equation*}
        I=\int_0^\infty\dd x~e^{-\epsilon x}e^{ikx}=\left.\frac{e^{(i(k-\epsilon))}}{ik-\epsilon}\right|_0^\infty=\frac{i}{k+i\epsilon}=\mathcal{P}\frac{i}{k}+\pi\delta(k).
    \end{equation*}
\end{proof}
\end{document}