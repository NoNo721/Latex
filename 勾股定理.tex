\documentclass{beamer}
\usetheme{PaloAlto}		%使用beamer幻灯片主题
\usecolortheme{crane}	%使用暖色调

\usepackage[UTF8,noindent]{ctexcap}

%beamer导言区
\title{杂谈勾股定理}
\subtitle{数学史讲座之一}
\institute{九章学堂}
\author{张三}
\date{\today}
\subject{勾股定理}
\keywords{勾股定理,历史}

\newtheorem{thm}{定理}	%定理环境
\renewcommand\proofname{证明}	%证明环境

\logo{\includegraphics[scale=0.05]{logo.png}}	%logo

\begin{document}
	%标题页
	\begin{frame}
		\titlepage	
	\end{frame}

	%总目录页
	\begin{frame}
		\tableofcontents[pausesections] %目录在每一项后面暂停
	\end{frame}

	\section{勾股定理在古代} %分节
	
		%分节标题页

		\begin{frame}
			\sectionpage
		\end{frame}

		\begin{frame}
			\frametitle{古中国数学}
			\framesubtitle{定理的发现}
			\alert<2->{中国在3000多年前就知道勾股定理的概念,比古希腊更早一些}	%第二步之后高亮
			\pause %暂停

			\textbf<3->{《周髀算经》的记载:}	%第三步之后加粗
			\pause
			\begin{itemize}
				\item<+-| alert@+-> 公元前11世纪,商高答周公问:
				%<+->中加号代表\pause,减号代表从\pause之后的步骤
				\begin{quote}
					勾广三,股修四,径隅五。
				\end{quote}
				\item<+-> 又载公元前7--6世纪陈子答荣方问,表述了勾股定理的一般形式:
				\begin{quote}
					若求邪至日者,以日下为勾,日高为股,勾股各自乘,并而开方除之,得邪至日。
				\end{quote}
			\end{itemize}
		\end{frame}
	
	
	\section{勾股定理在现代}

		\begin{frame}
			\sectionpage
		\end{frame}

		%分节目录页,只显示当前分节
		\begin{frame}
			\tableofcontents[currentsection]
		\end{frame}

		 	\subsection{勾} %次级分节
		 		\begin{frame}
		 			Hellow, world!
		 		\end{frame}

		 	\subsection{股}
		 		\begin{frame}
		 			\begin{thm}[勾股定理]
		 				直角三角形斜边的平方等于两直角边的平方和
		 			\end{thm}
		 			\begin{block}{Blocktitle}
		 				这是一个block
		 			\end{block}
		 			\begin{proof}
		 				$c^2=a^2+b^2$
		 			\end{proof}
		 		\end{frame}

		 	\subsection{定理}
		 		\begin{frame}
		 			\subsectionpage
		 		\end{frame}

		 	\subsection{动态展示}
		 		\begin{frame}
		 			\onslide<1>{只有第一步}
		 			\transblindshorizontal
		 			%水平百叶窗切换效果,只能放在frame内,放入后该frame内所有动态效果改变
		 			%换行

		 			\onslide<2->{第2步及之后}

		 			\onslide<1,3>{第1,3两步}
		 			%可以发现onslide命令不显示的内容易燃占位
		 		\end{frame}
		 		\begin{frame}
		 			\only<1>{1}\only<2>{2}\only<3>{3}\only<4>{4}

		 			\onslide<5>  数完了
		 			%可以发现only命令在不显示的步骤没有额外的占位
		 		\end{frame}
	 
	 \section{参考文献}

	 	\begin{frame}{参考文献}
	 	\end{frame}

	
\end{document}